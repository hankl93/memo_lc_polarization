\clearpage
\section{BEPCII and BESIII detector}
\label{sec:detector}

The BESIII detector is an approximately cylindrically symmetric detector with 93\% coverage of the solid angle around the $e^+e^-$ interaction point (IP). The components of the apparatus, ordered by distance from the IP, are a 43-layer small-cell main drift chamber (MDC), a time of-flight (TOF) system based on plastic scintillators with two layers in the barrel region and one layer in the end-cap region (the end-cap TOF system has been updated with the multi-gap resistive plate chamber in 2014~\cite{Yang:2014pfa}), a 6240-cell CsI(Tl) crystal electromagnetic calorimeter (EMC), a superconducting solenoid magnet providing a 1.0 T magnetic field aligned with the beam axis, and resistive plate muon-counter layers interleaved with steel. The momentum resolution for charged tracks in the MDC is 0.5\% for transverse momenta with 1 $\gev$. The energy resolution in the EMC is 2.5\% in the barrel region and 5.0\% in the end-cap region for  1 GeV photons. For charged tracks, particle identification (PID) for charged tracks combines measurements of the energy deposit d$E$/d$x$ in MDC and flight time in TOF and forms likelihoods $\mathcal{L}(h)(h = K, \pi)$ for a hadron h hypothesis. More details about the BESIII detector are provided in Ref.~\cite{BESIII:2009fln}.
