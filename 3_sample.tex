\clearpage
\section{Data and MC samples}
\label{sec:sample}
This analysis uses a combined data sample with a integrated luminosity of 6.4 fb$^{-1}$ collected at 13 c.m. energy points from $\sqrt{s}=4.600~\gev$ to 4.951~$\gev$, which are reconstructed under BOSS version of 7.0.6 and 7.0.7. More details about these data samples are listed in Table~\ref{tab:datainfo}.

\begin{table}[htbp]
    \centering
    \caption{The c.m. energy ($E_{\rm cms}$), the integrated luminosity, and BOSS version of the data samples in this analysis~\cite{BESIII:2015qfd,BESIII:2015zbz,BESIII:2022dxl,BESIII:2022ulv}.}
    \begin{tabular}{cccc}\hline\hline
        Sample  &  $\sqrt{s}$ (MeV) & $\mathcal{L}_{\rm int}$ (pb$^{-1}$)  & BOSS version \\\hline
        4600    & 4599.53 $\pm$ 0.07 $\pm$ 0.74 & 586.9 $\pm$ 0.1 $\pm$ 3.9 & \multirow{7}{4em}{7.0.6}\\
        4612    & 4611.86 $\pm$ 0.12 $\pm$ 0.30 & 103.65 $\pm$ 0.05 $\pm$ 0.55 \\
        4626    & 4628.00 $\pm$ 0.06 $\pm$ 0.32 & 521.53 $\pm$ 0.11 $\pm$ 2.76 \\
        4640    & 4640.91 $\pm$ 0.06 $\pm$ 0.29 & 551.65 $\pm$ 0.12 $\pm$ 2.92 \\
        4660    & 4661.24 $\pm$ 0.06 $\pm$ 0.29 & 529.43 $\pm$ 0.12 $\pm$ 2.81\\
        4680    & 4681.92 $\pm$ 0.08 $\pm$ 0.29 & 1667.39 $\pm$ 0.21 $\pm$ 8.84\\
        4700    & 4698.82 $\pm$ 0.10 $\pm$ 0.36 & 535.54 $\pm$ 0.12 $\pm$ 2.84 \\\hline
        4740    & 4739.70 $\pm$ 0.20 $\pm$ 0.30 & 163.87 $\pm$ 0.07 $\pm$ 0.87 & \multirow{6}{4em}{7.0.7}\\
        4750    & 4750.05 $\pm$ 0.12 $\pm$ 0.29 & 366.55 $\pm$ 0.10 $\pm$ 1.94 \\
        4780    & 4780.54 $\pm$ 0.12 $\pm$ 0.30 & 511.47 $\pm$ 0.12 $\pm$ 2.71 \\
        4840    & 4843.07 $\pm$ 0.20 $\pm$ 0.31 & 525.16 $\pm$ 0.12 $\pm$ 2.78 \\
        4920    & 4918.02 $\pm$ 0.34 $\pm$ 0.34 & 207.82 $\pm$ 0.08 $\pm$ 1.10\\
        4950    & 4950.93 $\pm$ 0.36 $\pm$ 0.38 & 159.28 $\pm$ 0.07 $\pm$ 0.84\\
        \hline\hline
    \end{tabular}
    \label{tab:datainfo}
\end{table}

Simulated samples are produced using a \textsc{Geant4}-based~\cite{GEANT4:2002zbu} Monte Carlo (MC) packages, which includes the geometric descriptions of the BESIII detector~\cite{Liang:2009zzb,Huang:2022wuo} and detector response. The beam-energy spread and initial-state radiation (ISR) in the $e^+e^-$ annihilation are simulated with the \textsc{KKMC} generator~\cite{Jadach:2000ir} in the MC samples. The final-state radiation (FSR) from the charged particles are modeled using \textsc{PHOTOS}~\cite{Richter-Was:1992hxq}. Three kinds of MC samples are used in ths analysis as described below:
\begin{itemize}
    \item Cocktail MC samples:
    \begin{itemize}
        \item Inclusive $\lcp\lcm$ MC samples, which simulate the production of $\lcp\lcm$ pairs. The cross section of $e^+e^- \to \lcp\lcm$ process is based on the analysis of Ref~\cite{Wang:memo}, and the subsequent decay modes of $\lcp(\lcm)$ are inclusive using the branching fractions (BFs) taken from the PDG~\cite{Workman:2022ynf}. The integrated luminosity is 40 times larger than that of data. These samples are used to study the background contribution from $\lcp\lcm$ pair production in data. 
        \item Inclusive hadron MC samples, which includes open-charmed mesons, ISR production of vector charmonium(-like) states and continuum processes. The known decay modes are modelled with \textsc{Evtgen}~\cite{Lange:2001uf,Ping:2008zz} using the BFs taken from the PDG~\cite{Workman:2022ynf}. The remaining unknown decays are simulated with \textsc{LUNDCHARM}~\cite{Chen:2000tv,Yang:2014vra}. The integrated luminosity is 10 times larger than that of data. These samples are used to study the background contribution from non-$\lcp\lcm$ pair production in data. 
    \end{itemize}
    \item Invisible PHSP signal MC samples, in which the signal process $\lcp \to pK^-\pi^+$ is generated with a uniform phase-space (PHSP) distribution and $\lcm$ decays into invisible particle which does not interact with the detector. 200K signal MC events are generated at each energy point, except for 400K events at $\sqrt{s} = 4.682~\gev$ due to its large integrated luminosity. The invisible signal MC samples are used to extract the pure signal shape, which will be used in the fit to the $\mbc$ distributions.
    \item Inclusive PHSP signal MC samples are produced with the same decay mode for the $\lcp$ in the signal side. For the other $\lcm$, it is required to inclusively decay to all possible final states with the BFs and models following the official DECAY.DEC file in the BesEvtGen package. The numbers of produced events are the same as the invisible signal MC samples. The inclusive signal MC samples are used in the MC integration part to construct the likelihood of the amplitude analysis with detection efficiencies automatically taken into account. 
    %The effects of MC statistics are studied and found to be small in the amplitude analysis, documented in Appendix~\ref{app:mc_statistics}.
\end{itemize}
